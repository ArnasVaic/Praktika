\section{Rezultatai, išvados ir pasiūlymai}

\subsection{Darbo rezultatai ir išvados}

\subsubsection*{Pagrindiniai rezultatai}

\begin{itemize}
    \item Išanalizuoti taisyklių rašytojų poreikiai ir identifikuoti dabartinių įrankių trūkumai
    \item Išanalizuoti galimi technologiniai sprendimai ir pasirinktas tinkamiausias -- dalykinės srities kalbos kompiliatorius su sintaksės paryškinimo įskiepiu skirtu konkrečiam teksto redaktoriui
    \item Suprojektuotas ir įgyvendinti du įrankiai -- kompiliatorius ir įskiepis
    \item Paruošta techninė dokumentacija, kurioje aprašyti įrankių veikimo principai bei dalykinės srities kalbos sintaksė ir semantika
    \item Paruošta diegimo infrastruktūra ir galutinės įrankių versijos sukeltos į aplinkas
\end{itemize}

\subsubsection*{Išvados}

\begin{itemize}
    \item Dalykinės srities kalbos kompiliatorius kartu su teksto redaktoriaus \textit{VSCode} įskiepiais visumoje sudaro lankstų sprendimą, kuris leidžia nesunkiai prisitaikyti prie vis besikeičiančių taisyklių rašytojų poreikių
    \item Sukurta įrankiu pora leidžia efektyviai dirbti su išaugusiu taisyklių kiekiu ir prisitaiko prie identifikuotų taisyklių rašytojų poreikių todėl galima teigti, kad darbo tikslas \enquote{sukurti sprendimą taisyklių rašymo problemai} buvo pasiektas
\end{itemize}

\subsection{Praktikos darbo privalumai ir trūkumai}

\subsubsection*{Privalumai}
\begin{itemize}
    \item Siekiant pagrindinio praktikos tikslo buvo susidurta su visais programų sistemų kūrimo etapais -- analize, projektavimu, įgyvendinimu, testavimu, diegimu, palaikymu bei techninės dokumentacijos ruošimu
    \item Kadangi įrankio poreikis ir problemos buvo tikros, įgyta patirtis yra vertinga ir prisidės prie tolimesnių projektų sėkmės užtikrinimo
\end{itemize}

\subsubsection*{Trūkumai}

Trūkumai nebuvo identifikuoti.
