\sectionnonum{Įvadas}

\subsection*{Praktikos vieta}

Profesinės praktikos vieta -- UAB „Admoneo“. Autorius šioje įmonėje dirba programuotoju, todėl būtent ši vieta buvo pasirinkta profesinei praktikai atlikti.

\subsection*{Problematika}

Buhalteriniai skaičiavimai dažnai remiasi Darbo kodeksu ir kitomis valstybės nustatytomis tvarkomis, pagal kurias skaičiuojamas darbo užmokestis, mokesčiai ir kt. Šios taisyklės turi daug ypatumų, dėl kurių buvo nuspręsta nenaudoti egzistuojančių programavimo kalbų joms įgyvendinti. Jų formalizavimui buvo sukurti specialiai pritaikyti įrankiai, leidžiantys taisykles aprašyti deklaratyviu stiliumi.

Ilgainiui nusistovėjo paprastas procesas, kaip šios taisyklės yra rašomos, palaikomos ir versijuojamos. Šiuo metu taisyklės rašomos viename \textit{Excel} tipo faile, kuris versijuojamas saugant jo kopijas \textit{OneDrive} diske, o failų varduose nurodomos pokyčių datos. Didėjant taisyklių apimčiai ir atsiradus poreikiui dirbti prie jų keliems žmonėms vienu metu, tapo akivaizdu, kad dabartiniai įrankiai ir procesas nėra pritaikyti tolimesniam darbui.

\subsection*{Praktikos tikslas ir uždaviniai}

Pagrindinis šios praktikos tikslas – sukurti sprendimą, leidžiantį keliems rašytojams efektyviai dirbti su taisyklių rinkiniu, jį paraleliai keisti, versijuoti naudojant tam pritaikytas technologijas ir tokiu būdu modernizuoti taisyklių rašymo procesą. 

\begin{itemize}
    \item Išanalizuoti esančius poreikius dalykinės srities kalbai

    \item Iš egzistuojančių kalbos elementų suprojektuoti dalykinės srities kalbą - jos formalią gramatiką, semantines taisykles bei moduliarumo palaikymą
    
    \item Suprojektuoti dalykinės srities kalbos kompiliatorių

    \item Įgyvendinti dalykinės srities kompiliatorių ir VSCode programai pritaikytą įskiepį sintaksės paryškinimui

    \item Užtikrinti kompiliatoriaus veikimo korektiškumą bei palengvinti kodo priežiūrą su automatinių
    testų pagalba

    \item Užtikrinti sėkmingą sukurtų įrankių diegimą

\end{itemize}

\newpage

\subsection*{Praktinės veiklos planas ir atlikimo eiga}

\begin{figure}[!htb]
    \centering
    \includegraphics[width=0.25\textwidth]{diagrams/flow.png}
    \caption{Praktinės veiklos planas}
    \label{fig:plan}
\end{figure}